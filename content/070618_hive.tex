\section{07.06.2018 - R / Hive}
- Suchen Sie sich einen für Sie interessanten Bereich aus bspw. Marktforschung, soziale Netzwerke, Wetterstationen, Gesundheitswesen, Natur und Sozialwissenschaften, etc.
- Erzeugen Sie für Ihr Beispiel zwei Hive Tabellen
- Füllen Sie die Tabellen mit einigen Testdaten (LOAD DATA, Insert).
- Bauen Sie eine R Verbindung zur HIVE auf.
- Analysieren Sie die Daten

\subsection*{Kurzdarstellung der Aufgabenstellung}
Mittels dem Anlegen von zwei Tabellen in HIVE und dem anschließenden Bearbeiten/Analysieren mit R soll die das Zusammenspiel von relevanter Software für BigData geübt/demonstriert werden.
\subsection*{Lösung}
\subsubsection*{HIVE}

- Ordner in /usr/tmp/hive/ anlegen

- Datensets „hotels.csv“ und „autos.csv“  in /usr/tmp/hive/ legen

- HIVE über den Terminal in Oracle VM4.9:
\begin{lstlisting}
HIVE
\end{lstlisting}

- separate Datenbank im HIVE anlegen, falls keine namentlich identische existiert:
\begin{lstlisting}
CREATE DATABASE IF NOT EXISTS diim18;
\end{lstlisting}

-Tabelle „hotels“ anlegen mit folgendem Befehl:
\begin{lstlisting}
Create TABLE hotels(gewinn FLOAT, preisInMio FLOAT,qm FLOAT,stadt String)       ROW Format DELIMITED FIELDS Terminated by ',' Lines terminated by '\n';
\end{lstlisting}

-Tabelle „autos“ anlegen mit folgendem Befehl:
\begin{lstlisting}
Create TABLE autos(preis FLOAT, registrierungJahr INT,ps FLOAT,km INT,  modell String, kraftstoff STRING, name STRING) ROW Format DELIMITED FIELDS      Terminated by ',' Lines terminated by '\n';
\end{lstlisting}

- Daten aus hotels.csv in HIVE Tabelle hotels laden:
\begin{lstlisting}
LOAD DATA LOCAL INPATH '/usr/tmp/hive/hotels/hotels.csv' Overwrite INTO         Table hotels;
\end{lstlisting}

- Daten aus hotels.csv in HIVE Tabelle hotels laden:
\begin{lstlisting}
LOAD DATA LOCAL INPATH '/usr/tmp/hive/autos/autos.csv' Overwrite INTO   Table autos;
\end{lstlisting}
- SCREENSHOT VON den angelegten Tabellen ggf interessant

=> Die Daten können nun mit SQL Statements betrachtet und manipuliert werden

\subsubsection*{R}

- starten von R durch Eingabe in den terminal auf der Oracle 4.9 VM:
\begin{lstlisting}
R
\end{lstlisting}

- installieren des RJDBC packages:
\begin{lstlisting}
install.packages("RJDBC",dep=TRUE)
\end{lstlisting}

- aufrufen des RJDBC packages:
\begin{lstlisting}
library("RJDBC")
\end{lstlisting}

- driver auf variable in R zur Verwendung zuweisen:
\begin{lstlisting}
drv <-JDBC("org.apache.hive.jdbc.HiveDriver","/usr/lib/hive/lib         /hive-jdbc.jar")
\end{lstlisting}

- Hierzu ich weiss nicht wozu man das folgende Braucht aber muss ja rein:

\begin{lstlisting}
library("ORCH")
ore.connect(type="HIVE")
ore.sync()
\end{lstlisting}

- Verbindung zu R aufbauen
\begin{lstlisting}
conn <- dbConnect(drv, "jdbc:hive2://localhost:10000/diim18", "", "")
\end{lstlisting}

- Inhalt der Tabelle autos im Hive auf die Variable autos in R zuweisen:
\begin{lstlisting}
autos <- dbGetQuery(conn, "Select * from autos")
\end{lstlisting}

- Ausgabe der ersten 5 Zeilen
\begin{lstlisting}
head(autos)
\end{lstlisting}

- lineares Regressionsmodell (lRm) erstellen: autos.preis = a * autos.km + b
\begin{lstlisting}
model <-lm(autos.preis~autos.km, data = autos)
\end{lstlisting}

- Modell Zusammenfassung ausgeben lassen (SCREEN-SHOT print(model))


- Erstellen einer neuen Spalte in Autos die den Preis auf Basis des lRm:
\begin{lstlisting}
autos$predicted_1 <- predict(model, autos)
\end{lstlisting}

- multiples Rm (mRm) erstellen: autos.preis = a * autos.km + b * autos.ps + b
\begin{lstlisting}
model <-lm(autos.preis~autos.km+autos.ps, data = autos)
\end{lstlisting}

- Modell Zusammenfassung ausgeben lassen (SCREEN-SHOT print(model))

- Erstellen einer neuen Spalte in Autos die den Preis auf Basis des mRm:
\begin{lstlisting}
autos$predicted_2 <- predict(model, autos)
\end{lstlisting}

print(model) => fehlt noch die Ausgabe der einzelnen Models
head(autos)

- Inhalt der Tabelle autos im Hive auf die Variable autos in R zuweisen:
\begin{lstlisting}
hotels <- dbGetQuery(conn, "Select * from hotels")
\end{lstlisting}

- Ausgabe der ersten 5 Zeilen
\begin{lstlisting}
head(hotels)
\end{lstlisting}

- lineares Regressionsmodell (lRm) erstellen: autos.preis = a * autos.km + b
\begin{lstlisting}
model <-lm(hotels.preisinmio~hotels.gewinn+hotels.qm, data = hotels)
\end{lstlisting}

- Erstellen einer neuen Spalte in Autos die den Preis auf Basis des lRm:
\begin{lstlisting}
hotels$predicted_1 <- predict(model, hotels)
\end{lstlisting}

- Modell Zusammenfassung ausgeben lassen (SCREEN-SHOT print(model))


- multiples Rm (mRm) erstellen: autos.preis = a * autos.km + b * autos.ps + b
\begin{lstlisting}
model <-lm(hotels.preisinmio~hotels.gewinn+hotels.qm+hotels.stadt, data =       hotels)
\end{lstlisting}

- Modell Zusammenfassung ausgeben lassen (SCREEN-SHOT print(model))


- Erstellen einer neuen Spalte in Autos die den Preis auf Basis des mRm:
\begin{lstlisting}
hotels$predicted\_2 <- predict(model, hotels)
\end{lstlisting}


\subsection*{Aufteilung der Aufgaben im Team}
\subsection*{Darstellung der benutzen Werkzeuge und Systeme}
\subsubsection*{Entwurfswerkzeug}
- HIVE (weil da die Daten strukturiert wurden?)

\subsubsection*{Entwicklungsumgebung}
- R

