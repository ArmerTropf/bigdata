\section{ARIS (14.06.2018) }
Definieren Sie für Ihr fiktives/reales Unternehmen folgende Diagrammtypen: 

Datenmodell, Geschäftsprozess, IT-Infrastruktur, Organigramm, Prozesslandschaft, BPMN, Whiteboard Beispiel, Systemlandschaft. Die Beispiele finden Sie im Elli.

\subsection*{Kurzdarstellung der Aufgabenstellung}
Die Darstellung von Prozessen soll mit ARIS anhand eines fiktiven Unternehmens erprobt werden. Hierfür sollen ein Datenmodell, Geschäftsprozessmodell, IT-Infrastruktur-Modell Organigramm, Prozesslandschafts-Modell, BPMN, Whiteboard-Beispiel und Systemlandschafts-Modell ausgearbeitet in ARIS ausgearbeitet werden.

\subsection*{Lösung}

\begin{itemize}
\item[-] Installation ARIS
\item[-] Starten ARIS
\item[-] Diagramme für den entsprechenden Aufgabenbereich Auswählen
\end{itemize}

\begin{enumerate}
\item Organigramm

Es wurde auf die Darstellung der einzelnen Personen mit Namen weitestgehend verzichtet, stellvertretend sind die Rollen („Leiter“, „MAs“ und „Azubi“) zu verstehen. Wobei es in jeder Organisationseinheit („Einkauf“, „Marketing“, etc.) einen Leiter und mehrere Mitarbeiter/Azubis geben kann. Da das Unternehmen auf Grund seiner Größe und des Aktionsfeldes nur im Kaufmännischen Bereich ausbilden kann, gibt es in der IT Abteilung keinen Azubi.

\item Prozesslandschaft
Die an der Wertschöpfung beteiligten Prozesse im folgenden Diagramm dargestellt: 

\item Geschäftsprozess
Der Geschäftsprozess des PSA Handels (keine eigene Produktion) für das fiktive Unternehmen ist im Folgenden dargestellt.

\item Datenmodell
In diesem Modell werden die Entitäten mit Ihren Attributen dargestellt für den vorherigen Geschäftsprozess bzw. die Prozesslandschaft.

\item BPMN
Die Arbeitsprozesse bezogen auf Angebot und Verkauf ist im Folgenden dargestellt. 

\item Whiteboard
Ziel des Brainstorming ist die Effizienzsteigerung mit folgenden Maßnahmen:
\begin{itemize}
	\item[-] Lieferantenangebotsübernahme in ERP mittels OCR
	\item[-] Webshop-Konzept mit Chat-Fkt.
	\item[-] Predictive-Analytice für Bestands-/Bedarfsplanung
\end{itemize}

\item IT-Infrastrukturmodell
Die Infrastruktur soll als Übersicht dienen und ist hier im kleinen Umfang dargestellt.
\item IT-Systemlandschaftsmodell
Die Systemlandschaft zeigt alle relevanten Systeme der fiktiven Firma
\end{enumerate}
\subsection*{Aufteilung der Aufgaben im Team}
Alle Aufgabenpunkte wurden gemeinsam bearbeitet.
\subsection*{Darstellung der benutzen Werkzeuge und Systeme}
\subsubsection*{Entwurfswerkzeug}
ARIS
\subsubsection*{Entwicklungsumgebung}
ARIS
