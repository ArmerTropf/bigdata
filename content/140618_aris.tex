\section{ARIS (14.06.2018) }
Definieren Sie für Ihr fiktives/reales Unternehmen folgende Diagrammtypen: 

Datenmodell, Geschäftsprozess, IT-Infrastruktur, Organigramm, Prozesslandschaft, BPMN, Whiteboard Beispiel, Systemlandschaft. Die Beispiele finden Sie im Elli.

\subsection*{Kurzdarstellung der Aufgabenstellung}
Die Darstellung von Prozessen soll mit ARIS anhand eines fiktiven Unternehmens erprobt werden. Hierfür sollen ein Datenmodell, Geschäftsprozessmodell, IT-Infrastruktur-Modell Organigramm, Prozesslandschafts-Modell, BPMN, Whiteboard-Beispiel und Systemlandschafts-Modell ausgearbeitet in ARIS ausgearbeitet werden.

\subsection*{Lösung}

\begin{itemize}
\item[-] Installation ARIS
\item[-] Starten ARIS
\item[-] Diagramme für den entsprechenden Aufgabenbereich Auswählen
\end{itemize}

\begin{enumerate}
\item Organigramm

Es wurde auf die Darstellung der einzelnen Personen mit Namen weitestgehend verzichtet, stellvertretend sind die Rollen („Leiter“, „MAs“ und „Azubi“) zu verstehen. Wobei es in jeder Organisationseinheit („Einkauf“, „Marketing“, etc.) einen Leiter und mehrere Mitarbeiter/Azubis geben kann. Da das Unternehmen auf Grund seiner Größe und des Aktionsfeldes nur im Kaufmännischen Bereich ausbilden kann, gibt es in der IT Abteilung keinen Azubi.

\textbf{Alle Diagramme befinden sich im Anhang des Zip-Files, da diese für das Dokument zu groß sind.}

\item Prozesslandschaft

Die an der Wertschöpfung beteiligten Prozesse: 

\item Geschäftsprozess

Der Geschäftsprozess des PSA Handels (keine eigene Produktion) für das fiktive Unternehmen.

\item Datenmodell

In diesem Modell werden die Entitäten mit Ihren Attributen dargestellt für den vorherigen Geschäftsprozess bzw. die Prozesslandschaft.

\item BPMN

Die Arbeitsprozesse bezogen auf Angebot und Verkauf. 

\item Whiteboard

Dieses Diagramm wurde zur Zusammenfassung eines Brainstormings mit dem Gesamtziel das Unternehmen digital auszurichten bzw. zu optimieren durch Digitalisierung. Hauptziel ist hierbei die Steigerung von Flexibilität, Senkung der Kosten und Einsatz von Predictive Ansätzen/Methoden. Dieses Ziel soll durch folgende Stages sind hierbei als relevante Punkte identifiziert worden:
\begin{itemize}
\item[-] Digitalisierung der bis dato manuellen Lieferantenangebotserfassung durch ein Webinterface für Lieferanten und eine KI die Angebote in Pdf oder Faxform „verarbeitet“ und im ERP bereitstellt
\item[-] Aufsetzen von einem Webshop mit Beratungskommunikations-Interface um die schnelllebige Kundenschaft befriedigen zu können
\item[-] Einsatz von Predictive Analytics (PA)im Bestandsmanagement (Bedarfsplanung) sowie die Potenzialerforschung von Externen Daten („PA“ Kompetenz aufbauen)
\end{itemize}

\item IT-Infrastrukturmodell

Die Infrastruktur soll als Übersicht dienen und ist hier im kleinen Umfang dargestellt.
\item IT-Systemlandschaftsmodell

Die Systemlandschaft zeigt alle relevanten Systeme der fiktiven Firma.
\end{enumerate}
\subsection*{Aufteilung der Aufgaben im Team}
Alle Aufgabenpunkte wurden gemeinsam bearbeitet.
\subsection*{Darstellung der benutzen Werkzeuge und Systeme}
\subsubsection*{Entwurfswerkzeug}
ARIS
\subsubsection*{Entwicklungsumgebung}
ARIS
