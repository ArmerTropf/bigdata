\section{R RJDBC / Access (31.05.2018) }
\begin{itemize}
\item[-]Suchen Sie sich einen für Sie interessanten Bereich aus bspw. Marktforschung, soziale Netzwerke, Wetterstationen, Gesundheitswesen, Natur und Sozialwissenschaften, etc.
\item[-]Führen Sie einen Tabellen Entwurf durch.
\item[-]Erzeugen Sie die Tabellen (Oracle Datenbank).
\item[-]Füllen Sie die Tabellen mit einigen Testdaten
\item[-]Bauen Sie eine RJDBC Verbindung (R) zur Oracle Datenbank auf.
\item[-]Analysieren Sie die Daten
\item[-]Suchen Sie in deren Datenbeständen nach unbekannten Zusammenhängen.
\end{itemize}
\subsection*{Kurzdarstellung der Aufgabenstellung}
Es sollen Daten aus einem beliegen Bereich gewählt werden und in einer Datenbank abgelegt werden. Auf die Datenbank soll mit R zugriffen werden um die Daten analysieren zu können.
\subsection*{Lösung}
\begin{itemize}
\item[-]Gewählter Datensatz: https://www.kaggle.com/harlfoxem/housesalesprediction beinhaltet Hausverkaufsdaten von Häusern in King County (USA) im Jahre 2014
\item[-]Daten wurden in MS-Access (C:/users/admin/documents/DATA\_DB.mdb) mit der CSV-Importfunktion eingelesen(\autoref{fig:rjdbc1})
\begin{figure}[!htb]
        \begin{minipage}{1\textwidth}
                \centering
                \includegraphics[width=0.90\textwidth]{pics/rjdbc1.png}\par\vspace{0cm}
                \caption{Import MS-Access}
                \label{fig:rjdbc1}
        \end{minipage}
\end{figure}

Für die analyse wurde nur R gestartet und folgende Aktionen/Befehle ausgeführt:
\item[-]Installieren RODBC Package:
\begin{lstlisting}
install.packages("RODBC")
\end{lstlisting}
\item[-]Aufrufen Package RODBC:
\begin{lstlisting}
library(RODBC)
\end{lstlisting}
\item[-]Zuweisen Genutzten Datenbank auf (abgelegt in R source Ordner)
\begin{lstlisting}
channel <- odbcConnectAccess("DATA\_DB.mdb")
\end{lstlisting}

\item[-]Zweisen der Datentabelle auf Objekt:
\begin{lstlisting}
data = sqlQuery(channel ,paste('SELECT * FROM HOUSE\_PRICE\_SQRFT'))
\end{lstlisting}

\item[-]Anzeigen der ersten sechs Zeilen der Tabelle/Objekt(\autoref{fig:rjdbc2})
\begin{figure}[!htb]
        \begin{minipage}{1\textwidth}
                \centering
                \includegraphics[width=0.90\textwidth]{pics/rjdbc2.png}\par\vspace{0cm}
                \caption{Anzeigen der Zeilen}
                \label{fig:rjdbc2}
        \end{minipage}
\end{figure}

\item[-]Analyse Histogram Preis(\autoref{fig:rjdbc3}):
\begin{lstlisting}
hist(data$price,xlab = "price",col = "green",border = "red"
\end{lstlisting}

\begin{figure}[!htb]
        \begin{minipage}{1\textwidth}
                \centering
                \includegraphics[width=0.90\textwidth]{pics/rjdbc3.png}\par\vspace{0cm}
                \caption{Analyse Histogramm}
                \label{fig:rjdbc3}
        \end{minipage}
\end{figure}

\item[-]Datensatz auf Preise bis maximal 1mio reduzieren:
\begin{lstlisting}
data = sqlQuery(channel ,paste('SELECT * FROM HOUSE_PRICE_SQRFT WHERE PRICE < 1000001'))
\end{lstlisting}

\item[-]Neues Histogramm ausführen(\autoref{fig:rjdbc4}):
\begin{lstlisting}
hist(data$price,xlab = "price",col = "green",border = "red")
\end{lstlisting}

\begin{figure}[!htb]
        \begin{minipage}{1\textwidth}
                \centering
                \includegraphics[width=0.90\textwidth]{pics/rjdbc4.png}\par\vspace{0cm}
                \caption{Weiteres Histogramm}
                \label{fig:rjdbc4}
        \end{minipage}
\end{figure}

\item[-]Abfrage minimum Preis in der Datei(\autoref{fig:rjdbc4-1}):
\begin{figure}[!htb]
        \begin{minipage}{1\textwidth}
                \centering
                \includegraphics[width=0.50\textwidth]{pics/rjdbc4-1.png}\par\vspace{0cm}
                \caption{Abfrage Preis}
                \label{fig:rjdbc4-1}
        \end{minipage}
\end{figure}

\item[-]Boxplot (Wohnraufgrößen Variation pro Baujahr)(\autoref{fig:rjdbc5}):
\begin{lstlisting}
boxplot(sqft_living ~ yr_built, data = data, xlab = "YEAR BUILT",+    ylab = "SQFT_LIVINV", main = "SQFT PER YEAR OF CONSTRUCTION")
\end{lstlisting}

\begin{figure}[!htb]
        \begin{minipage}{1\textwidth}
                \centering
                \includegraphics[width=0.50\textwidth]{pics/rjdbc5.png}\par\vspace{0cm}
                \caption{Boxplot SQFT per year of construction}
                \label{fig:rjdbc5}
        \end{minipage}
\end{figure}

\item[-]Erstellen von SCATTER PLOT MATRIX(\autoref{fig:rjdbc6}):
\begin{lstlisting}
pairs(~price+sqft_living+sqft_lot,data = data,  +    main = "Scatterplot Matrix")
\end{lstlisting}

\begin{figure}[!htb]
        \begin{minipage}{1\textwidth}
                \centering
                \includegraphics[width=0.50\textwidth]{pics/rjdbc6.png}\par\vspace{0cm}
                \caption{Scatterplot}
                \label{fig:rjdbc6}
        \end{minipage}
\end{figure}

\item[-]multiple lineare Regression (Price in Abängigkeit Sqft\_Living und Sqft\_Loft)

\item[-]Installation und Öffnung der relevanten Daten
\begin{lstlisting}
install.packages("caret")
install.packages("ggplot2")
install.packages("lattice")
library(caret)
\end{lstlisting}

\item[-]Verwendung der Daten in der variable „data“(\autoref{fig:rjdbc7})

\begin{figure}[!htb]
        \begin{minipage}{1\textwidth}
                \centering
                \includegraphics[width=0.90\textwidth]{pics/rjdbc7.png}\par\vspace{0cm}
                \caption{Variablenverwendung}
                \label{fig:rjdbc7}
        \end{minipage}
\end{figure}

\item[-]Verwendung Modell für multiple lineare Regression mit allen mathematisch        vorstellbar relevanten spalten
\begin{lstlisting}
model <-lm(price~bedrooms+bathrooms+sqft_living+sqft_lot+waterfront+    view+condition+grade+sqft_above+sqft_basement+yr_built+yr_renovated,    data = data)
\end{lstlisting}

\item[-]Ausgabe des berechneten Modells(\autoref{fig:rjdbc8}):
\begin{figure}[!htb]
        \begin{minipage}{1\textwidth}
                \centering
                \includegraphics[width=0.60\textwidth]{pics/rjdbc8.png}\par\vspace{0cm}
                \caption{Ausgabe Berechnungsmodell}
                \label{fig:rjdbc8}
        \end{minipage}
\end{figure}
\end{itemize}
\subsection*{Aufteilung der Aufgaben im Team}
Alle Aufgabenpunkte wurden gemeinsam bearbeitet
\subsection*{Darstellung der benutzen Werkzeuge und Systeme}
\subsubsection*{Entwurfswerkzeug}
MS-ACCESS
\subsubsection*{Entwicklungsumgebung}
R (32bit Version da MS-ACCESS 32bit Version ansonsten Verbindung in der Form nicht möglich)

